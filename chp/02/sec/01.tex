\section{Ambientes virtuais}\label{sec:ambientes-virtuais}

A primeira coisa a se fazer é certificar-se de que seja possível criar ambientes
virtuais que possam isolar as dependências da aplicação em desenvolvimento do
restante do sistema. Imagine, por exemplo, que você esteja utilizando uma certa
biblioteca para a linguagem Python, numa versão \(X\) em seu trabalho, numa
versão \(Y\) para um trabalho da faculdade, e, finalmente, numa versão \(Z\),
mais recente do que as outras, para acompanhar o desenvolvimento de tal
biblioteca. Eis a pergunta que surge: como manter tantas versões de uma mesma
biblioteca em um único sistema? A resposta é simples: \textit{ambientes
virtuais}.

\subsection{Instalando o pip}\label{ssec:instalando-o-pip}

\href{https://pypi.org/project/pip/}{\texttt{pip}} significa \textit{package
installer for Python} e você pode usá-lo para instalar pacotes disponíveis no
\href{https://pypi.org/}{PyPi}, que significa \textit{Python Package Index}.
Existe mais de uma maneira de se instalar o \texttt{pip} em seu sistema. Aqui,
vamos instalá-lo através do gerenciador de pacotes em nosso sistema.

\begin{lstlisting}[language=bash,caption={Instalando o \texttt{pip} no Debian}]
  > sudo apt install python3-venv python3-pip
\end{lstlisting}

\begin{lstlisting}[language=bash,caption={Instalando o \texttt{pip} no Fedora}]
  > sudo dnf install python3-pip python3-wheel
\end{lstlisting}

\subsection{Criando ambientes virtuais}\label{ssec:criando-ambientes-virtuais}

A criação de ambientes virtuais é muito simples. A fim de sermos mais didáticos,
iremos supor que estejamos interessados em aprender a desenvolver sistemas
\textit{web} com \href{https://www.djangoproject.com/}{Django}. Para começar,
vamos criar uma pasta chamada \textit{projects}, que irá conter todos os
projetos desenvolvidos neste computador. O comando para isto é o \texttt{mkdir}:

\begin{lstlisting}[language=bash,caption={Criando o diretório de projetos}]
  > mkdir -p projects
\end{lstlisting}

Feito isto, vamos mudar para o diretório recém criado e, então, criaremos um
novo diretório, digamos, \textit{meu\_primeiro\_projeto\_django}, para abrigar o
desenvolvimento da nossa aplicação. Fica assim:

\begin{lstlisting}[language=bash,caption={Criando o diretório que abrigará nossa aplicação}]
  > cd projects
  > mkdir -p meu_primeiro_projeto_django
\end{lstlisting}

A seguir, adentramos o diretório recém criado, \textbf{criamos} e
\textbf{ativamos} um ambiente virtual chamado \texttt{venv} (pouco criativo,
certo?).

\begin{lstlisting}[language=bash,caption={Criando o ambiente virtual}]
  > cd meu_primeiro_projeto_django
  > python3 -m venv venv
\end{lstlisting}

A opção \(-m\) é como se evoca um módulo de maneira não interativa em Python. O
primeiro \texttt{venv} é, portanto, o módulo de mesmo nome em Python enquanto
que o segundo \texttt{venv} é o nome escolhido para o ambiente virtual em
questão. Procedendo assim, dá-se o nome de \texttt{venv} para todos os ambientes
virtuais existentes na máquina. Mas, então, não há o risco de confusão? Não, não
há. Lembre-se de que o \texttt{venv} recém criado reside na pasta chamada
\textit{meu\_primeiro\_projeto\_django}.

Agora, é hora de ativar o ambiente virtual e instalar as dependências para o
nosso projeto, neste caso Django. A maneira de se fazer isto é com o comando
\texttt{source}.

\begin{lstlisting}[language=bash,caption={Ativando o ambiente virtual -- parte 1}]
  > source venv/bin/activate
\end{lstlisting}

Enquanto isto irá funcionar quando o \textit{shell} utilizado for o
\textbf{bash}, não podemos afirmar com certeza que todas as pessoas no planeta
estarão a utilizá-lo, não é mesmo? E quando o shell em questão for o excelente
\textbf{ksh}, por exemplo? Substitua o comando \texttt{source} por um ponto:

\begin{lstlisting}[language=bash,caption={Ativando o ambiente virtual -- parte 2}]
  > . venv/bin/activate
\end{lstlisting}

Feito isto, vamos finalmente instalar Django:

\begin{lstlisting}
  > pip install django
\end{lstlisting}

Resumindo:

\begin{enumerate}
  \item
    Criamos uma pasta para abrigar a nossa aplicação;
  \item
    Entramos nesta pasta e, então, criamos um ambiente virtual para instalar as
    dependências do projeto de maneira isolada do restante do sistema;
  \item
    Ativamos o ambiente virtual criado no passo anterior e, por fim,
  \item
    Instalamos as dependências.
\end{enumerate}

São muitos passos, não? Além disso, muitos dirão que estamos no século XXI, que
usar o terminal é coisa do passado e, principalmente, que tem-se que digitar
muita coisa! Prezados, \textit{makefile}'s são nossos amigos! Como qualquer
explicação sobre o \href{https://www.gnu.org/software/make/}{GNU Make} escapa
aos propósitos iniciais destas notas, gostaríamos apenas de dizer que
\texttt{make} pode ser utilizado para automatizar o processo de se transformar
códigos em produtos. Em verdade, esta explicação não faz juz ao incrível poder
de \texttt{make}. Por exemplo, eu o utilizei para criar o PDF que você lê agora!

Para dar um singelo exemplo, digamos que o nosso projeto utilize, além do já
mencionado Django, o excelente
\href{https://github.com/psf/black}{\texttt{black}}. Nós podemos, então, criar
um arquivo chamado \textit{requirements.txt} na raiz de nosso projeto e com ele
listar as dependências em nosso projeto, uma por linha. Ficaria assim:

\begin{lstlisting}[caption={Um exemplo de requirements.txt}]
  black
  django
\end{lstlisting}

Feito isto, poderíamos criar um \texttt{makefile}, também na raiz de nosso
projeto, com a seguinte regra:

\begin{lstlisting}[caption={Exemplo de makefile}]
  ready:
    python3 -m venv venv; \
    . venv/bin/activate; \
    pip install -U pip; \
    pip install -r requirements.txt; \
    deactivate
\end{lstlisting}

Deste modo, quando quisermos criar um ambiente virtual e com ele instalar as
dependências de nosso projeto, bastaria agora digitar um único comando:

\begin{lstlisting}[language=bash,caption={Reduzindo-se a quantidade de comandos digitados}]
  > make ready
\end{lstlisting}

A esta altura alguns poderiam perguntar:

Mas, eu precisaria criar um makefile com uma regra como esta para cada novo
projeto que eu iniciar?

Boa pergunta! Sim, isto é verdade. No meu caso, tenho
\href{https://github.com/SirVer/ultisnips}{snippets} de códigos em
\href{https://www.vim.org/}{meu editor de textos favorito}, que automatizam o
processo de criação de makefiles (e tantas outras coisas) nas linguagens de
programação que mais utilizo, como C, Python e \LaTeX{} (e, sim, eu penso em
\LaTeX{} como uma linguagem de programação; o que difere neste caso é o
propósito do programa que estou a escrever).

Eu não quero causar a impressão equivocada de que o \texttt{make} apenas serve
para se reduzir a quantidade de comandos digitados ao terminal, pois este é um
efeito do seu uso. O GNU Make é muito eficiente em detectar o que não foi
modificado desde a última compilação e, portanto, economiza recursos da máquina
e reduz tempo de execução de modo que em grandes projetos o seu uso se torna
obrigatório (ou, então, de alguma de suas alternativas).

\subsection{Outras referências}\label{ssec:outras-referências}

Leituras sugeridas:

\begin{enumerate}
  \item
    \href{https://python.org/}{https://python.org/}
  \item
    \href{https://docs.python-guide.org/}{O guia do mochileiro para Python}
  \item
    \href{https://developers.google.com/edu/python/}{Google's Python Class}
  \item
    Para se gerar documentação de softwares:
    \href{https://docs.readthedocs.io/en/stable/intro/getting-started-with-sphinx.html}{Sphinx}
\end{enumerate}
