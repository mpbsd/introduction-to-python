\chapter{Instalação}\label{chp:instalação}

Muitos sistemas são entregues ao usuário final com alguma versão da linguagem
\python{} pré-instalada. Você pode tentar executar o comando \texttt{python},
que inicia uma sessão interativa do interpretador de comandos da linguagem, e,
assim, checar se o mesmo já se encontra instalado.

A seguir cobriremos uma maneira de se instalar a linguagem em sistemas Linux,
tanto \debiangnulinux{} quanto \fedoraproject. Se  você é um usuário do
Microsoft Windows, por favor considere a adoção de um sistema de código aberto.
Até lá, recomendamos o uso do
\href{https://learn.microsoft.com/en-us/windows/wsl/about}{WSL} (Windows
Subsystem for Linux). No macOS, pode-se usar o gerenciador de pacotes
\href{https://brew.sh}{Homebrew}.

\section{Instalação}\label{sec:instalação}

Esta seção descreve a instalação de Python em sistemas excelentes sistemas
operacionais de código aberto: Debian GNU/Linux Fedora Linux.
 % linux systems
