\section{Distribuições Linux}\label{sec:distribuições-linux}

Em sistemas Linux os softwares são, geralmente, chamados de \textit{pacotes},
distribuídos através de \textit{repositórios} e gerenciados por meio de
\textit{gerenciadores de pacotes}. Cada distribuição Linux possui seu próprio
gerenciador de pacotes (e, sim, o \slackwarelinux{} possui o
\href{https://docs.slackware.com/slackware:package_management}{\texttt{pkgtools}}).
Em caso de dúvida, consulte a documentação para o seu sistema.

\section{Debian GNU/Linux}\label{ssec:debian-gnu/linux}

No Debian, vamos utilizar o gerenciador de pacotes \texttt{apt}.

\begin{lstlisting}[language=bash,caption={Instalando Python 3 no Debian},label={install-python3-debian}]
  > sudo apt install python3
\end{lstlisting}

\section{Fedora Linux}\label{ssec:fedora-linux}

No Fedora, vamos utilizar o gerenciador de pacotes \texttt{dnf}.

\begin{lstlisting}[language=bash,caption={Instalando Python 3 no Fedora}]
  > sudo dnf install python3
\end{lstlisting}
